\section{Introduction to the non-LTE Radiation Code}

The NLTE (non-Local Thermodynamic Equalibrium) radiation scheme is an adjustment to heating rates and fluxes to account for NLTE effects in the radiation scheme. These effects become important above~$70\, \text{km}$, and should allow the UM to be extended to above~$85\, \text{km}$.

The code was written following the papers by Dr Victor Fomichev: the relevant papers are listed below:

\begin{enumerate}
\item Fomichev, V. I., Blanchet J.-P., and Turner D. S. (1998), Matrix
parameterization of the 15 $\mu$m CO$_2$ band cooling in the middle
and upper atmosphere for variable CO$_2$ concentration,
J. Geophys. Res., 103, No. D10, 11505 - 11528
 
\item Fomichev, V. I., and Blanchet J.-P. (1995), Development of the new
CCC/GCM radiation model for extension into the Middle Atmosphere,
Atmosphere-Ocean, 33, No. 3, 513-529.
 
\item Ogibalov V. P., and V. I. Fomichev (2003), Parameterization of
solar heating by the near IR CO2 bands in the mesosphere, Adv. Space
Res., 32, No. 5, 759-764.
 
\item Fomichev V. I., V. P. Ogibalov, and S. R. Beagley (2004), Solar
heating by the near-IR CO2 bands in the mesosphere,
Geophys. Res. Lett., 31, L21102, \\
\noindent doi:10.1029/2004GL020324.
\end{enumerate}

Reference 1 (Fomichev et al., 1988) is for the~CO$_2$ cooling code. Reference 2 (Fomichev and Blanchet, 1995) includes~O$_3$ cooling with the~CO$_2$ code. References 3 and 4 (Ogibalov and Fomichev, 2003; Fomichev et al., 2004) describe a scheme for~CO$_2$ heating in the near-IR (near-Infra-Red) bands, which is most important in the mesosphere, and require an NLTE treatment.

The NLTE code lives in the SOCRATES code under {\tt srcs/nlte/}. {\tt nlte\_heating\_mod.f90} is the LW (longwave) correction  and {\tt sw\_nlte\_heating\_mod.f90} is the SW (shortwave) correction.

The UM calls the LW NLTE heating rate correction from \\
\noindent {\tt src/atmosphere/radiation\_control/lw\_rad.f90}. \\
\noindent The SW code {\tt src/atmosphere/radiation\_control/sw\_rad.f90} calls a series of three subroutines: the NLTE heating rate correction is done in the last of these:\\
\noindent {\tt src/atmosphere/radiation\_control/socrates\_postproc.f90}.

\section{LW Schemes}

Here,~$x$ is a dimensionless vertical scale height.

\paragraph{Fomichev's 1998 scheme for CO$_2$}
\begin{enumerate}
\setcounter{enumi}{-1}
\item $x \leq 2$ \\
  Existing radiation scheme used.
\item  $2 < x \leq 10$ \\
   CO$_2$ volume mixing ratio ($c_{CO_2}$) independent of height. \\ 
   Cooling only proportional to $T$ and $c_{CO_2}$. \\
   A matrix parameterisation is used for this LTE layer.
\item  $10 < x \leq 12.5$ \\
   CO$_2$ volume mixing ratio independent of height. \\ 
   There is no significant influence from atomic O$_2$, the matrix 
   parametisation is used with the coefficients corrected.
\item  $12.5 \leq x < 14$ \\ 
   CO$_2$ volume mixing ratio proportional to scale height $x$. \\
   The recurrence formula is used, corrected such that the escape 
   function is corrected to account for radiative heat exchange with
   the atmosphere and contribution of other bands other than the 
   fundamental band.
\item  $14 \leq x \leq 16.5$ \\ 
   CO$_2$ volume mixing ratio proportional to scale height. \\
   The fundamental CO$_2$ isotope dominates the radiative cooling. The 
   recurrence formula is used for the NLTE layer.
\item  $x > 16.5$ \\ 
   Smooth transition to cooling-to-space approximation, without any 
   parameterisation.
\end{enumerate}

\paragraph{Fomichev's 1995 scheme for O$_3$}
\begin{enumerate}
\setcounter{enumi}{-1}
\item  $x \leq 2$ \\
   Existing radiation scheme used.
\item  $2 < x \leq 10.5$ \\
   Matrix parameterisation used with corrected components.
\item  $x > 10.5$ \\
   No contribution.
\end{enumerate}

\section{Full LW code outline}

\subsection{Set up vertical scale height ($x$) and corresponding 
  pressure levels}

   $x = \ln \frac{100000}{p_x}$ \\

   \paragraph{Outputs}
   \begin{enumerate}
   \item $x$: Dimensionless scale height, set in code from 
     $0 \leq x \leq 17.5$ with steps of $\Delta x = 0.25$ (bottom of atmosphere 
     to top)
   \item $p_x$: Pressure corresponding to the x-levels $x$ (Pa)
   \end{enumerate}

   \paragraph{References}
   \begin{itemize}
   \item Equation - Fomichev 98 eq 1 and Fomichev 95 eq 1 (but differ
     in terms of log vs ln)
   \item Scale height should be ln not log - \\
     \url{http://scienceworld.wolfram.com/physics/PressureScaleHeight.html}
   \item $\Delta x$ intervals - Fomichev 98 section 2
   \end{itemize}

\subsection{Interpolate the UM temperatures and gas mix ratios to $x$}

   \paragraph{Inputs}
   \begin{enumerate}
   \item $p$: Pressure levels, from the UM (top of atmosphere to 
     bottom) (Pa)
   \item $T$: Temperatures corresponding to the pressure levels, 
     from the UM (top of atmosphere to bottom) (K)
   \item $g_q$: Mass mixing ratio at different pressure levels 
     and for different gases q, from the UM (top of atmosphere to bottom)
     (parts per million, ppm)
   \end{enumerate}

   \paragraph{Outputs}
   \begin{enumerate}
   \item $T_x$: Temperatures corresponding to the dimensionless scale $x$, 
     (bottom of atmosphere to top) (K)
   \item $g_{x,q}$: Mass mixing ratios for different gases q corresponding 
     to the dimensionless scale $x$ (bottom of atmosphere to top) (ppm)
   \end{enumerate}

   \paragraph{Requirements}
   \begin{itemize}   
   \item Spline interpolation code - using {\tt interpolate\_p} in 
     SOCRATES.
   \end{itemize}

\subsection{Convert mass mixing ratios to volume mixing ratios over 
  scale required}

   $c_{x,q} = \frac{\mu_x}{M_q}g_{x,q}$ \\

   \paragraph{Inputs}
   \begin{enumerate}
   \item $g_{x,q}$: Mass mixing ratio (ppm)
   \item $\mu_x$: Molar mass of dry air per scale height (kg mol$^{-1}$)
   \item $M_q$: Molar mass of required atmospheric species (kg mol$^{-1}$)
   \end{enumerate}

   \paragraph{Outputs}
   \begin{enumerate}
   \item $c_{x,q}$: Volume mixing ratios for different gases q corresponding 
     to the dimensionless scale $x$ (parts per million volume, ppmv)
   \end{enumerate}

   \paragraph{Data}
   \begin{itemize}
   \item Currently there are no mass mixing ratios for O$_2$, N$_2$
     and O in the UM, and these are taken from Fomichev's code. These
     profiles should be replaced with something that can be
     referenced. The Air Force Geophysics Laboratory paper (see below)
     has O$_2$ data only. Fomichev lists his O$_2$, N$_2$ and O data
     as all being in ppmv, but he then converts O$_2$ and N$_2$ data
     as though it was in ppm. Is this a bug, or are these all in ppm?
     I assume Fomichev's data is in ppmv as labelled and do not convert.
   \item Molar mass of dry air per scale height currently taken from 
     Fomichev's code, need proper data from somewhere that can be 
     referenced. (This doesn't seem to be defined in atm as defined in 
     {\tt def\_atm.f90} - this only has atm\%mass which is column mass per layer). 
   \item For molar masses of required atmospheric species, the best
     reference would be \textit{General Inorganic Chemistry} by
     J. A. Duffy (1970), Longmans (as used in the Edwards-Slingo
     radiation code). Couldn't get hold of this, so currently using
     values from \\
     \url{https://environmentalchemistry.com/yogi/reference/molar.html})
    \end{itemize}

   \paragraph{References}
   \begin{itemize}
   \item Method - Fomichev 98, section 6, paragraph 1. 
   \item For conversion of mass mixing ratio to volume mixing ratio see e.g. \\
     \url{https://software.ecmwf.int/wiki/pages/viewpage.action?pageId=61121586}
   \item Possible O$_2$ volume mixing ratio data - AFGL Atmospheric
     Constituent Profiles (0-120km), AFGL-TR-86-0110, Environmental
     Research Papers No. 954
   \end{itemize}

   \paragraph{Notes}
   \begin{itemize}
   \item The use in the Fomichev code of $M_O$ to calculate $c_{x,O_2}$ 
     instead of $M_{O_2}$ is probably just a bug according to Omar.
   \item Fomichev uses the info that the mass mixing ratio of CO$_2$
     is near constant below $x = 12.5$, but I have currently used the 
     mmr of CO$_2$ directly from the UM code.
   \item Another possible bug in the Fomichev code, {\tt o2\_conv} is
     calculated as {\tt am\_in(ii)} in the~$x = 51, 67$ range which appears to be
     the wrong airmass.
   \end{itemize}


\subsection{Calculate the exponential part of the Planck function for 
    CO$_2$ and O$_3$}

   $\phi_{x,q} = \exp(-hc_sV_q/kT_x)$ \\

   \paragraph{Inputs}
   \begin{enumerate}
   \item $h$: Planck's constant (J s)
   \item $c_s$: Speed of light (m s$^{-1}$)
   \item $V_{CO_2}$, $V_{O_3}$: Frequencies of the main vibrational 
     transitions of the main isotopes of CO$_2$ and O$_3$ (m$^{-1}$)
   \item $k$: Boltzmann's constant (J K$^{-1}$)
   \item $T_x$: Temperature per scale height $x$ (K)
   \end{enumerate}

   \paragraph{Outputs}
   \begin{enumerate}
   \item $\phi_{x,CO_2}, \phi_{x,O_3}$: Exponential parts of the Planck function 
     for CO$_2$ and O$_3$ respectively (dimensionless)
   \end{enumerate}

   \paragraph{Data}
   \begin{itemize}
   \item Currently using data for the main vibrational transitions of the 
     main isotopes of CO$_2$ and O$_3$ directly from Fomichev's code - 
     could not find a proper reference for these.
   \end{itemize}

   \paragraph{References}
   \begin{itemize}
   \item Equation - Fomichev 98 equation 6 (shows T-dependence
     only). The exponent constant is $hc_sV_{CO_2}/k$, but I don't
     have a reference for this. There is a small difference in the
     value of the exponent calculated from the constants and that
     given in the paper.
   \end{itemize}


\subsection{Calculate the CO$_2$ coefficients for the matrix 
  parameterisation $2 < x < 12.5$ for the particular concentration 
  of CO$_2$}

   \begin{itemize}
   \item For each given scale height $x$, height level $j$ and concentration 
     $c$, the parameters $a_{j,x}$ and $b_{j,x}$ are found by 
     interpolating over $c$ between sets of values $a_{c,j,x}$ and 
     $b_{c,j,x}$.
   \item For those sets of parameters where $a_{c,j,x}$ or $b_{c,j,x}$ have
     the same sign across all $c$, linear interpolation of 
     log[$a_{c,j,x}/c_{x,q}$] or log[$b_{c,j,x}/c_{x,q}$] is recommended. 
     Otherwise a second-order interpolation of $a_{c,j,x}$ or $b_{c,j,x}$ 
     should be used.
   \end{itemize}   
   
   Note that Fomichev calculates heat capacity in units of cm$^2$s$^{-3}$,
   which is equivalent to 10$^4$Jkg$^{-1}$s$^{-1}$. I leave the coefficients
   in these units but need to convert the heat capacity to standard units
   of Jkg$^{-1}$s$^{-1}$ once calculated.

   \paragraph{Inputs}
   \begin{enumerate}
   \item $a_{c,j,x}$, $b_{c,j,x}$: Tables of matrix coefficients for CO$_2$ 
     versus given volume mixing ratios of CO$_2$
   \item $c_{x<12.5,CO_2}$: Volume mixing ratio of CO$_2$ for x$<$12.5 (ppmv) 
   \end{enumerate}

   \paragraph{Outputs}
   \begin{enumerate}
   \item $a_{j,x}$, $b_{j,x}$: Matrix coefficients interpolated to the 
     required CO$_2$ concentration
   \end{enumerate}

   \paragraph{Requirements}
   \begin{itemize}   
   \item Requires linear and third-order spline interpolation. 
   \end{itemize}

   \paragraph{References}
   \begin{itemize}
   \item Fomichev 98, section 6, end paragraph 2 for notes on how to 
     interpolate
   \item Fomichev 98, tables 2-9 for CO$_2$ coefficient tables. The
     tables in Fomichev 98 differ from the code by a smallish but
     significant amount.
   \item Fomichev 98, section 3.1 paragraph 3 suggests 360ppm is used 
     for the concentration of CO$_2$, but this is outdated.  \\
     \url{http://ds.data.jma.go.jp/gmd/wdcgg/pub/products/summary/sum41/10_03co2.pdf}
     for the CO$_2$ concentration below $x=12.5$. 2015 had 400.0ppmv of 
     CO$_2$, with +2.1ppmv each year, i.e. 406.3ppmv in 2018. (Note that
     this paper refers to mole fraction, which is equivalent to volume
     mixing ratio.)
   \end{itemize}

   \paragraph{Issues}
   \begin{itemize}
   \item Fomichev suggests a CO$_2$ concentration of 360ppm in his paper, but
     in his code he uses 0.72e-3. His paper is also unclear on whether the 
     coefficients correspond to volume or mass mixing ratios - he refers
     to volume mixing ratios, but gives the units ppm. I suspect they are
     all supposed to be volume mixing ratios.
   \item Is {\tt interp\_p} okay to use?
   \item Note that all matrix constants must be divided by their reference
     CO$_2$ concentration and multiplied by the CO$_2$ concentration below 
     $x=12.5$, this isn't 100\% clear from the paper.
   \end{itemize}


\subsection{Calculation of heating rates using matrix approach}

   $$\epsilon_x = \sum_{j=-5}^{3}[a_{j,x} + b_{j,x}\phi_{x,CO_2}]\phi_{j,CO_2}
   + \sum_{j=-3}^{2} c_{O_3}c_{j,x}\phi_{j,O_3}$$

   where:
   
   Dimensionless height steps for CO$_2$:
   \begin{table}[H]
     \begin{tabular}{ll}\hline
       \textbf{$j$} & \textbf{$\Delta x_j$} \\
       \hline
       -5           & -6.25                 \\
       -4           & -3.0                  \\
       -3           & -1.75                 \\
       -2           & -0.75                 \\
       -1           & -0.25                 \\
       0            & 0                     \\
       1            & 0.25                  \\
       2            & 0.75                  \\
       3            & 1.5                   \\
       \hline
     \end{tabular}
   \end{table}

   Dimensionless height steps for O$_3$:
   \begin{table}[H]
     \begin{tabular}{ll}\hline
       \textbf{$j$} & \textbf{$\Delta x_j$} \\
       \hline
       -3           & -4.0                 \\
       -2           & -1.25                 \\
       -1           & -0.25                 \\
       0            & 0                     \\
       1            & 0.25                  \\
       2            & 1.75                  \\
       \hline
     \end{tabular}
   \end{table}

   \begin{itemize}  
   \item This applies from $2 < x < 10.5$ (CO$_2$ and O$_3$) and 
     $10.75 < x \leq 12.5$ (CO$_2$ only)
   \end{itemize}   
   
   \paragraph{Inputs}
   \begin{enumerate}
   \item $a_{j,x}$, $b_{j,x}$, $c_{j,x}$: Matrix coefficients
   \item $\phi_{x,CO_2}$, $\phi_{x,O_3}$: Exponential part of Planck function 
     for CO$_2$ and O$_3$ (dimensionless)
   \item $\Delta x_j$ versus $j$: dimensionless height steps and indices 
     of levels used to calculate heat exchange 
   \end{enumerate}

   \paragraph{Outputs}
   \begin{enumerate}
   \item $\epsilon_x$: Heating rates up to $x \leq 12.5$ (J kg$^{-1}$ s$^{-1}$)
   \end{enumerate}

   \paragraph{References}
   \begin{itemize}
   \item Fomichev 98 equation 5 and Fomichev 95 equation 2, for the summing
     equation
   \item Fomichev 98 section 4.1 final paragraph for notes on the 
     parameterised grid
   \item Fomichev 98 table 1 and Fomichev 95 table 1 for the 
     parameterised grid steps
   \end{itemize}


\subsection{Calculate the CO$_2$ column amount for $12.5 < x < 16.5$}

   The integral over the depth of the atmosphere defines the \emph{atmospheric column} of CO$_2$ (or another gas species~$X$) as
   %
   \begin{equation}
   u_{x, \text{CO}_2}\ =\ \int n_{x, \text{CO}_2} dx, \nonumber
   \end{equation}

   \noindent where~$x$ is the dimensionless scale height and~$n_{x, \text{CO}_2}$ is the number density of CO$_2$ (molecules cm$^{-3}$). The number density~$n_{x, \text{CO}_2}$ and the volume mixing ratio~$c_{x, \text{CO}_2}$ are related by the number density of air~$n_{x, a}$ as follows:
   %
   \begin{equation}
   n_{x, \text{CO}_2}\ =\ c_{x, \text{CO}_2} n_{x,a}. \nonumber
   \end{equation}

   \noindent The number density of air~$n_{x,a}$ is related to the number of moles of air~$N$ and the volume~$V$ by Avogadro's constant~${A_v = 6.023 \times 10^{23}\, \text{molecules}\, \text{mol}^{-1}}$:
   %
   \begin{equation}
   n_{x,a}\ =\ \dfrac{A_v N}{V}.\nonumber
   \end{equation}

   \noindent The ideal gas law may be expressed as~${pV = NRT}$, which can be used to obtain the following expression for the number density of air:
   %
   \begin{equation}
   n_{x,a}\ =\ \dfrac{A_v p}{RT}\nonumber
   \end{equation}

   \noindent From the ideal gas law, it can also be obtained that~${\rho_{x,a} = p\mu_x / RT}$, where~$\mu_x$ is the molar mass of dry air per scale height (kg$\,$mol$^{-1}$). By assuming hydrostatic balance,~${\frac{dp}{dx} = - \rho_{x,a}\, g}$, together these give:
   %
   \begin{equation}
   dx\ =\ \dfrac{RT}{p \mu_x\, g} dp. \nonumber
   \end{equation}

   With these expressions, the following equation for the CO$_2$ column amount can be obtained:
   %
   \begin{alignat}{4}
   u_{x, \text{CO}_2}\ & =\ && \int n_{x, \text{CO}_2}\, dx \nonumber \\
                       & =\ && \int c_{x, \text{CO}_2}\, n_{x,a}\, dx \nonumber \\
                       & =\ && \int c_{x, \text{CO}_2} \dfrac{A_v p}{RT} \dfrac{RT}{p\, m_{\text{air}}\,g} dp \nonumber \\
                       & =\ && \dfrac{A_v}{g} \int \dfrac{c_{x, \text{CO}_2}}{\mu_x}\, dp. \nonumber
   \end{alignat}

   Next, differentiate this expression with respect to the dimensionless scale height~$x$. Note that~${p_x = 10^5 e^{-x}}$, and so~${dp = -10^5 e^{-x}}$. Then:
%
   \begin{alignat}{4}
   \dfrac{d u_{x, \text{CO}_2}}{dx}\ & =\ && \dfrac{A_v}{g} \dfrac{d}{dx} \bigg( \int \dfrac{c_{x, \text{CO}_2}}{\mu_x} dp \bigg) \nonumber \\
                                     & =\ && \dfrac{A_v}{g} \dfrac{d}{dx} \bigg[ \int \dfrac{c_{x, \text{CO}_2}}{\mu_x} (- 10^5 e^{-x})\, dx \bigg] \nonumber \\
                                     & =\ && - \dfrac{10^5 A_v}{g} \dfrac{c_{x, \text{CO}_2}\, e^{-x}}{\mu_x}.
   \end{alignat}

%   Now in a discrete form, we may obtain the following expressions:
   %
%   \begin{alignat}{4}
%   \Delta x\ & =\ && 0.25, \nonumber \\
%   \Delta e^{-x}\ & =\ && -e^{-x}\Delta x.
%   \end{alignat}

   To evaluate the expression for~$\Delta u_x$: the difference between~$u_x$ at~$x_i$ and~$x_{i - 1}$, we need the average value of~$e^{-x}$ evaluated at~$x_i$ and~$x_{i - 1}$, so $e^{-x}|_{i - 1/2} = \frac{1}{2} (e^{-x_i} + e^{-x_{i - 1}})$. Note that~$\Delta x = 0.25$. Then the expression for the derivative of the column amount may be expressed as:
   %
   \begin{alignat}{4}
   \Delta u_{x, \text{CO}_2}|_{i - 1/2}\ & =\ && - \dfrac{10^5 A_v}{g} \bigg( \dfrac{c_{x,\text{CO}_2}}{\mu_x} e^{-x} \Delta x \bigg)_{i - 1/2} \nonumber \\
                                         & =\ && \dfrac{0.25 \times 10^5 A_v}{g} \bigg( \dfrac{c_{x_i, \text{CO}_2} e^{-x_i}}{\mu_{x_i}} + \dfrac{c_{x_{i - 1}, \text{CO}_2} e^{-x_{i - 1}}}{\mu_{x_{i-1}}} \bigg). \nonumber
   \end{alignat}

   \noindent Now let~$Y = 0.25 \times 10^5 A_v / 2g $ and rearrange this to obtain:
%
   \begin{equation}
   u_{x_{i - 1}}\ =\ u_{x_{i}} + Y \bigg( \dfrac{c_{x_i, \text{CO}_2} e^{-x_i}}{\mu_{x_i}} + \dfrac{c_{x_{i - 1}, \text{CO}_2} e^{-x_{i - 1}}}{\mu_{x_{i-1}}} \bigg) \nonumber
   \end{equation}

   This formulation is then identical to what is used in Fomichev's stand-alone code.

   The column amount above $x = 16.5$, $u_{x=16.5,CO_2}$, is used as a 
   boundary condition to generate the array of $u_{x,CO_2}$ for 
   $12.5 < x < 16.5$.

   \paragraph{Inputs}
   \begin{enumerate}
   \item $A_v$: Avogadro's constant (mol$^{-1}$)
   \item $\mu_x$: Molar mass of dry air per scale height (kg mol$^{-1}$)
   \item $g$: Acceleration due to gravity (ms$^{-1}$)
   \item $c_{x,CO_2}$: Volume mixing ratio for CO$_2$ (ppmv)
   \item $x$: Dimensionless scale height (dimensionless)
   \item $u_{x>16.5,CO_2}$: CO$_2$ column amount above $x = 16.5$ (cm$^{-2}$)
   \end{enumerate}

   \paragraph{Outputs}
   \begin{enumerate}
   \item $u_{x,CO_2}$: Column amount, or number of molecules of CO$_2$ per 
     unit area (cm$^{-2}$)
   \end{enumerate}

   \paragraph{References}
   \begin{itemize}
   \item Fomichev 98 section 6 paragraph 4 states the CO$_2$ column amount 
     is used, but not how it is calculated or what the value is.
   \item `Introduction to Atmospheric Chemistry', Daniel J. Jacob, Princeton University Press, 1999. Available at \\ 
\url{http://acmg.seas.harvard.edu/people/faculty/djj/book/}. \\
This is used to derive the equation for the CO$_2$ column amount.
   \end{itemize}

   \paragraph{Issues}
   \begin{itemize}
   \item Where does the CO$_2$ column amount above $x = 16.5$ come from? No value 
      seems to be given in the paper. Set in code, but not explained, just 
      with a note that says "does this need changing?" (written by Omar)
   \item Are SI units correct in column calc, e.g. p in Pa?
   \end{itemize}


\subsection{Calculate the coefficients $d_j$ for the recurrence formula 
    for $12.5 < x < 16.5$}

   $d_x = \alpha_{u,x}L_u$ for $12.5 \leq x \leq 13.75$

   \noindent $d_x = L_u$ for $x >= 14.0$
   
   \noindent where\\
   \noindent $L_u$ is the escape function\\
   \noindent $\alpha_{u,x}$ is the correction to the escape function 

   \begin{itemize}   
   \item $L_u$ for $x$ is found by linear interpolation of 
     log $L_u$ over $u_x$.
   \item $\alpha_x$ for $x$ is found by linear interpolation of 
     log $\alpha_{u,x}$ over $u_x$, where a different log $\alpha_{u,x}-u_x$
     association is available for each $x$.
   \item Note that Tables 10 and 11 in Fomichev 98 have $\log(u)$ rather 
     than $u$, so these have been converted.
   \item For the range $12.5 \leq x \leq 13.75$, $\log L_u + \log \alpha_{u,x}$
     is calculated before the exponential is taken to obtain $d_x$ 
   \end{itemize}
  
   \paragraph{Inputs}
   \begin{enumerate}
   \item $u_x$: CO$_2$ column amount (cm$^{-2}$)
   \item $L_u$: Escape function for CO$_2$ column amount $u_x$ 
   \item $\alpha_{u,x}$: Correction to escape function for CO$_2$ column 
     amount $u_x$ and scale height $x$ 
   \end{enumerate}

   \paragraph{Outputs}
   \begin{enumerate}
   \item $d_x$: Coefficients for the recurrence formula (dimensionless)
   \end{enumerate}

   \paragraph{Requirements}
   \begin{itemize}   
   \item Linear interpolation. 
   \end{itemize}   

   \paragraph{References}
   \begin{itemize}
   \item Fomichev 98 table 10 - Escape function $L_u$ for CO$_2$ column 
     amount $u_x$
   \item Fomichev 98 table 11 - Correction to escape function $\alpha_{u,x}$ 
     for CO$_2$ column amount $u_x$ and for $x$ 
   \item Fomichev 98 section 6 paragraph 4 for interpolation methods.
   \end{itemize}

   \paragraph{Issues}
   \begin{itemize}
   \item Fomichev 98 tables 10 and 11 are not identical to the code - for
     a start, the code has u rather than log(u) and $\alpha$ rather than
     $\log \alpha$.
   \end{itemize}


\subsection{Calculate number density}

   $\rho_x = \frac{p_x}{kT_x}$

   \noindent Use
   
   \noindent $x = \ln \frac{1000}{p_x}$
   
   \noindent where $p_x$ is in mbar (and 1 mbar = 100 Pa = 100 kg m$^{-1}$ s$^{-2}$) to get
   
   \noindent $\rho_x = 10^5 \frac{e^{-x}}{kT_x}$
   
   \paragraph{Inputs}
   \begin{enumerate}
   \item $x$: Dimensionless scale height (dimensionless)
   \item $k$: Boltzmann Constant in (J K$^{-1}$)
   \item $T_x$: Temperature at scale height $x$ (K)
   \end{enumerate}

   \paragraph{Outputs}
   \begin{enumerate}
   \item $\rho_x$: Number density at scale height $x$ (m$^3$)
   \end{enumerate}

\subsection{Calculate parameter lambda for use in the recurrence 
  formula}
  
   $\lambda_x = A_{CO_2} / \{ A_{CO_2} + \rho_x[c_{x,N_2}k_{N_2} + c_{x,O_2}k_{O_2} + c_{x,O}k_o ] \}$

   \noindent where the collisional deactivation rate constants in m$^3$s$^{-1}$ are

   \noindent $k_{N_2} = 5.5 \times 10^{-17} \sqrt{T_x} + 6.7 \times 10^{-10} \exp(-83.8T_x^{-1/3})$ \\
   \noindent $k_{O_2} = 10^{-15} \exp(23.37 - 230.9T_x^{-1/3} + 564T_x^{-2/3})$ \\
   \noindent $k_O = 3 \times 10^{-12}$
   
   \noindent and where
   
   \noindent $A_{CO_2}$ is the Einstein A coefficient for CO$_2$ in s$^{-1}$
   
   \noindent $\rho_x$ is the number density in m$^{-3}$ of the background atmosphere

   \paragraph{Inputs}
   \begin{enumerate}
   \item $A_{CO_2}$: Einstein A coefficient for the fundamental band for the 
     main isotope of $CO_2$ (s$^{-1}$)
   \item $x$: Dimensionless scale height (dimensionless)
   \item $k$: Boltzmann Constant (J K$^{-1}$)
   \item $T_x$: Temperature at scale height $x$ (K)
   \item $k_{N_2}$, $k_{O_2}$, $k_O$: Collisional deactivation rate constants 
     (m$^3$ s$^{-1}$)
   \item $c_{x,N_2}$, $c_{x,O_2}$, $c_{x,O}$: Volume mixing ratios (ppmv)
   \end{enumerate}

   \paragraph{Outputs}
   \begin{enumerate}
   \item $\lambda_x$: Recurrence formula parameter (dimensionless)
   \end{enumerate}

   \paragraph{References}
   \begin{itemize}
   \item Fomichev 98 equation 8
   \item Fomichev 98 section 2.2 for the collisional deactivation rate 
     constants (taken from Shved et al, 1998)
   \end{itemize}

   \paragraph{Issues}
   \begin{itemize}
   \item The only place this constant is referred to as Einstein's A
     coefficient is in the code, not the paper. Is this actually this
     coeff, or is it a paper town?
   \item Where can I find Einstein's coefficient for the fundamental band 
     for CO$_2$ main isotope?
   \end{itemize}


\subsection{Calculation of the boundary condition at $x=12.5$ for the 
    recurrence formula}

   $\gamma_{x=12.5} = \dfrac{\epsilon_{x=12.5}\mu_{x=12.5}}{K c_{x=12.5,CO_2}[1 - \lambda_{x=12.5}]}$ \\

   \noindent where

   \noindent $K = N_ahc_sV_{CO_2}A_{CO_2}W$ in mol$^{-1}$m$^2$kgs$^{-3}$
   
   \noindent where $W$ is the ratio of the statistical weights for the fundamental transition.

   \paragraph{Inputs}
   \begin{enumerate}
   \item $N_a$: Avogadro's Constant (mol$^{-1}$)
   \item $h$: Planck Constant (m$^2$ kg s$^{-1}$)
   \item $c_s$: Speed of light (m s$^{-2}$)
   \item $V_{CO_2}$: Frequency of the main vibrational transition of CO$_2$ 
     (m$^{-1}$)
   \item $A_{CO_2}$ : Einstein A coefficient for the fundamental band for the 
     main isotope of $CO_2$ (s$^{-1}$)
   \item $W$: Ratio of the statistical weights for the fundamental transition 
     (dimensionless)
   \item $\mu_{x<12.5}$: Molar mass of dry air per scale height (kg mol$^{-1}$)
   \item $\epsilon_{x=12.5}$: Heating rate at x=12.5 calculated from the 
     matrix method (J kg$^{-1}$ s$^{-1}$)
   \item $c_{x,CO_2}$: Volume mixing ratio of CO$_2$ (ppmv)
   \item $\lambda$: Recurrence formula parameter (dimensionless) 
   \end{enumerate}

   \paragraph{Outputs}
   \begin{enumerate}
   \item $\gamma_{x=12.5}$: Boundary condition for the recurrence formula 
     at $x = 12.5$ (dimensionless)
   \end{enumerate}

   \paragraph{References}
   \begin{itemize}
   \item Fomichev 98 equation 10
   \item Fomichev 98 section 4.1 paragraph 5 for frequency of the main
     transition 01$^1$0-00$^0$0 of the ${}^{12}$C$^{16}$O$_2$ isotope, 
     667.3799 cm$^{-1}$
   \end{itemize}

   \paragraph{Issues}
   \begin{itemize}
   \item Where can I find Einstein's coefficient for the fundamental band for 
     CO$_2$ main isotope?
   \item The constant for Fomichev equation 10 does not match the 1/constb in 
     the code, which it corresponds to. There is a note in the code about how 
     to calculate this constant which does not seem to be in the paper.
   \end{itemize}

    
\subsection{Solve the recurrence formula for the heating rate 
  correction, and calculate the heating rate for $12.5 < x < 16.5$}

   Use

   \noindent $[1 - \lambda_x(1 - D_x)]\gamma_x = [1 - \lambda_{x-1}(1 - D_{x-1})]\gamma_{x-1} + D_{x-1}\phi_{x-1} - D_x\phi_x$

   \noindent where

   \noindent $D_x = \frac{1}{4}(d_{x-1} + 3d_x)$
   \noindent $D_{x-1} = \frac{1}{4}(3d_{x-1} + d_x)$

   \noindent to find the $\gamma_x$, using the boundary condition value $\gamma_{x=12.5}$. Then

   \noindent $\epsilon_x = \frac{K c_{x,CO_2}(1 - \lambda_x)}{\mu_x} \gamma_x$

   \paragraph{Inputs}
   \begin{enumerate}
   \item $d_x$: Coefficients for the recurrence formula (dimensionless)
   \item $\lambda_x$: Recurrence formula parameter (dimensionless)
   \item $\gamma_{x=12.5}$: Boundary condition for the recurrence formula 
     at $x = 12.5$ (dimensionless)
   \item $K$: Constant defined as before (mol$^{-1}$ m$^2$ kg s$^{-3}$)
   \item $c_{x,CO_2}$: Volume mixing ratio of CO$_2$ (dimensionless)
   \item $\mu_x$: Molar mass of dry air per scale height (kg mol$^{-1}$)
   \end{enumerate}

   \paragraph{Outputs}
   \begin{enumerate}
   \item $\epsilon_x$: Heating rate (J kg$^{-1}$ s$^{-1}$)
   \end{enumerate}

   \paragraph{References}
   \begin{itemize}
   \item Fomichev 98 equation 9
   \item Fomichev 98 equation 11
   \item Fomichev 98 equation 7
   \end{itemize}

   \paragraph{Issues}
   \begin{itemize}
   \item The constant for Fomichev equation 7 does not exactly match the 
     const in the code which it corresponds to. There is a note in the 
     code about how to calculate this constant which does not seem to be 
     in the paper - I need a reference for this.
   \end{itemize}

 
\subsection{Calculation of the heating rate above $x=16.5$}

   $\epsilon_{x>16.5} = \frac{K c_{x,CO_2}(1 - \lambda_x)}{\mu_x} [F_{x=16.5} - \phi_{x,CO_2}]$

   \noindent where

   \noindent $F$ is the upward flux, with $F_{x=16.5}$ defined by the boundary condition 

   \noindent $F_{x=16.5} = \gamma_{x=16.5} + \phi_{x=16.5,CO_2}$ 

   \paragraph{Inputs}
   \begin{enumerate}
   \item $K$: Constant defined as before (mol$^{-1}$ m$^2$ kg s$^{-3}$)
   \item $c_{x,CO_2}$: Volume mixing ratio of CO$_2$ (ppmv)
   \item $\lambda_x$: Recurrence formula parameter (dimensionless)
   \item $\mu_x$: Molar mass of dry air per scale height (kg mol$^{-1}$)
   \item $\gamma_{x=16.5}$: Boundary condition for the recurrence formula 
     at $x = 16.5$ (dimensionless)
   \item $\phi_{x,CO_2}$: Exponential part of the Planck function for CO$_2$
     (dimensionless)
   \end{enumerate}

   \paragraph{Outputs}
   \begin{enumerate}
   \item $\epsilon_x$: Heating rate for $x > 16.5$ (J kg$^{-1}$ s$^{-1}$)
   \end{enumerate}

   \paragraph{References}
   \begin{itemize}
   \item Fomichev 98 equation 13
   \item Fomichev 98 equation 14
   \end{itemize}


\subsection{Convert heating rates into K/day using the specific heat 
    capacity } 

   $\epsilon_x = \frac{nC_P\Delta T_x}{m_{total}}$

   \noindent where\\
   $\epsilon_x$ is the heating rate in J s$^{-1}$ kg$^{-1}$ \\
   $C_P$ is the specific heat capacity in J mol$^{-1}$ K$^{-1}$ \\
   $n$ is the number of moles in mol \\
   $\Delta T_x$ is the temperature change in K s$^{-1}$ \\
   $m_{total}$ is the total mass of gas in kg 

   Using $n = \frac{m_{total}}{\mu_x}$:

   \noindent $\Delta T_x = \frac{\epsilon_x \mu_x}{C_P}$.

   For a monatomic ideal gas

   \noindent $C_P = \frac{5}{2}R$.

   \noindent For diatomic and polyatomic ideal gases

   \noindent $C_P = \frac{7}{2}R$.
   
   \noindent where $R$ is the gas constant.

   For $x \leq 10.25$ there is assumed not to be any significant 
   concentration of monatonic gas.

   \noindent $C_P = \frac{7}{2}R$

   \noindent For $x \geq 10.25$ it is assumed that atomic oxygen makes a significant 
   contribution.

   \noindent $C_P = R[\frac{7}{2}(1 - c_{x,O}) + \frac{5}{2}c_{x,O}]$.

   Finally convert from units of K s$^{-1}$ to K day$^{-1}$.

   \paragraph{Inputs}
   \begin{enumerate}
   \item $\epsilon_x$: Heating rate (J s$^{-1}$ kg$^{-1}$)
   \item $\mu_x$: Molar mass of dry air per scale height (kg mol$^{-1}$)
   \item $R$: Gas constant (J mol$^{-1}$ K$^{-1}$)
   \item $c_O$: Volume mixing ratio for O (absolute units, not ppmv)
   \end{enumerate}

   \paragraph{Outputs}
   \begin{enumerate}
   \item $\Delta T_x$ in (K day$^{-1}$)
   \end{enumerate}

   \paragraph{References}
   \begin{itemize}
   \item None in papers. See e.g. \\
   \url{http://physics.bu.edu/~redner/211-sp06/class24/class24_heatcap.html}
   \end{itemize}

   \paragraph{Issues}
   \begin{itemize}
   \item What is a decent reference for the calculation of the heat capacity?
   \item Where is the reference that the O concentration should only be 
     considered at $x >= 10.5$? Is this simply from looking at the O profile?
   \end{itemize}


\subsection{Heating rates interpolated to the original pressure levels}

   \paragraph{Inputs}
   \begin{enumerate}
   \item $\epsilon_x$: Heating rates corresponding to the dimensionless 
     scale height $x$ (J kg$^{-1}$ s$^{-1}$)
   \item $p_x$: Pressure corresponding to the x-levels $x$ (converted
     to top of atmosphere to bottom) (Pa)
   \item $p$: Pressure levels, from the UM (top of atmosphere to bottom)
     (Pa)
   \end{enumerate}

   \paragraph{Outputs}
   \begin{enumerate}
   \item $\epsilon_p$: Heating rates corresponding to the original pressure 
     levels in the UM (top of atmosphere to bottom) (J kg$^{-1}$ s$^{-1}$)
   \end{enumerate}

   \paragraph{Requirements}
   \begin{itemize}   
   \item 3rd order spline interpolation
   \end{itemize}


\subsection{Heating rates blended with the LTE rates}

   For $p < 0.1$, $\epsilon = \epsilon_p$ \\
   For $p >= 0.1$, $\epsilon = \epsilon_{p,LTE}$

   \paragraph{Inputs}
   \begin{enumerate}
   \item $\epsilon_p$: Heating rates corresponding to the original pressure 
     levels in the UM (top of atmosphere to bottom) (J kg$^{-1}$ s$^{-1}$)
   \item $\epsilon_{p,LTE}$: Heating rates from the UM with no NLTE 
     correction applied (top of atmosphere to bottom) (J kg$^{-1}$ s$^{-1}$) 
   \item $p$: Pressure levels, from the UM (top of atmosphere to bottom)
     (Pa)
   \end{enumerate}

   \paragraph{Ouputs}
   \begin{enumerate}
   \item $\epsilon_{p,NLTE}$: Blended heating rates with NLTE correction 
     applied (top of atmosphere to bottom) (J kg$^{-1}$ s$^{-1}$)
   \end{enumerate}

   \paragraph{Issues}
   \begin{itemize}
   \item Is there a reference for the level at which the heating rates are 
     blended?
   \end{itemize}



\subsection{Misc}

I use NIST values for Physical constants: \\
\url{https://physics.nist.gov/cuu/Constants/} \\ 
CO$_2$ fundamental vibrational frequency - \\
    \url{https://webbook.nist.gov/cgi/cbook.cgi?ID=B4000020&Mask=800} \\
O$_3$ fundamental vibrational frequency - \\
   \url{https://webbook.nist.gov/cgi/formula?ID=B4000064&Mask=800}

O2 and N2 profiles are taken from the United States Standard Atmosphere
USSA-76 model (Chamberlain and Hunten 1987).

Might possibly be a useful reference on statistical weights - \\
\url{https://www.cfa.harvard.edu/atmosphere/publications/2006-EinsteinA-JQSRT-98.pdf}

James has confirmed that I need to loop over n\_profile (number of
atmospheric profiles for radiation calculations) in the code, but can
feel free to ignore n\_channel (number of output channels for
diagnostics)


Tables 1-4 in Fomichev 95 for $a_{O_3,j,x}$ and $b_{O_3,,j,x}$ differ
from those in the code - the tables in the paper are for 6 points in
the atmosphere, the ones in the code for 9 points.

\subsection{Outstanding Issues}

\begin{enumerate}
\item Some of the data is taken from the Fomichev code directly, and
  therefore doesn't have a good reference that can be given. Ideally 
  this should be replaced with referenceable data.
\item In nlte\_heating\_mod.F90, the gases to pick from the gas list
  sp\_lw\_ga7 are referenced directly by hardcoding the
  gas\_list\_index.  This should work with the data files so as not to
  be hardcoded.
\item Still issues calculating the $CO_2$ column constant.
\end{enumerate}

\section{SW Schemes}

\paragraph{Fomichev and Ogibalov's 2003/2004 scheme for CO$_2$}
\begin{enumerate}
\setcounter{enumi}{-1}

\item $x \leq 2$ \\
  Existing radiation scheme used.
\item  $2 < x \leq 14$ \\
   Parameterisation scheme used which sets heating adjustment according 
   to CO$_2$ volume mixing ratio ($c$) and solar zenith angle. \\ 
\end{enumerate}

\section{Full code outline SW}

\subsection{Notes}
\begin{enumerate}

\item This is stated in Ogibalov and Fomichev to be a daytime correction
  to the CO$_2$ cooling previously only parameterised for nighttime 
  conditions. Do I need to apply only partially?
\item The heating rate is blended to only $x=15.5$, should I do this? 
  (Probably, there is no parameterisation data about $x=14$)
\end{enumerate}


\subsection{Scale CO$_2$ column amount by the tropospheric CO$_2$ volume mixing ratio}

Scale the CO$_2$ column amount per dimensionless scale height from the
parameterisation data by the ratio of the tropospheric CO$_2$ volume
mixing ratio in the model, and that in the parameterisation data set.

$u_{CO_2, x} = u_{CO_2, x, ref} \times \dfrac{c_{CO_2, x=trop}}{c_{CO_2, x=trop, ref}}$

   \paragraph{Inputs}
   \begin{enumerate}
   \item $u_{CO_2, x, ref}$: CO$_2$ column amount by dimensionless scale height parameterisation data 
   \item $c_{CO_2, x=trop, ref}$: Tropospheric value of CO$_2$ volume mixing ratio from parameterisation data 
   \item $c_{CO_2, x=trop}$: Tropospheric value of CO$_2$ volume mixing ratio from model 
   \end{enumerate}

   \paragraph{Outputs}
   \begin{enumerate}
   \item $u_{CO_2, x}$: CO$_2$ column amount by dimensionless scale height data scaled by volume mixing ratio of tropospheric CO$_2$
   \end{enumerate}


\subsection{Calculate the amount of CO$_2$ along the zenith path}

   Mean effective pathlength factor

   \noindent $P = \frac{35}{\sqrt{1224\cos\theta_z^2 + 1}}$

   \noindent Scale the CO$_2$ column amount by the mean effective pathlength factor

   \noindent $u_{CO_2, x, z} = u_{CO_2, x} \times P$

   \paragraph{Inputs}
   \begin{enumerate}
   \item $u_{CO_2, x}$: CO$_2$ column amount by dimensionless scale height data scaled by volume mixing ratio of tropospheric CO$_2$
   \item $\cos\theta_{z}$: Cosine of the solar zenith angle
   \end{enumerate}

   \paragraph{Outputs}
   \begin{enumerate}
   \item $u_{CO_2, x, z}$: CO$_2$ column amount by dimensionless scale height data along the zenith path scaled by volume mixing ratio of tropospheric CO$_2$
   \end{enumerate}

   \paragraph{References}
   \begin{itemize}
   \item Rodgers C. D., 1967, The radiative heat budget of the troposphere and lower stratosphere
   \item Other refs, e.g. Li \& Shibata, 2006, On the Effective Solar Pathlength, \\ \url{https://journals.ametsoc.org/doi/pdf/10.1175/JAS3682.1}
   \end{itemize}

\subsection{User parameterisation to find heating rate}

   Interpolate the log of the CO$_2$ path amount linearly to find the heating 
   rate from the parameterisation.

   For values outside the range for which interpolation data is available: \\
   If $\log u_{CO_2, x, z} < \log u_{param, CO_2, x}(1)$, set $q_{CO_2} = q_{param, CO_2, x}(1)$ \\
   If $\log u_{CO_2, x, z} > \log u_{param, CO_2, x}(10)$, set $q_{CO_2} = 0$ \\

   \paragraph{Inputs}
   \begin{enumerate}
   \item $\log u_{CO_2, x, z}$: log CO$_2$ column amount along zenith path
   \item $\log u_{param, CO_2, x}$: Parameterisation reference data CO$_2$ column 
     amounts, ten values at each scale height
   \item $q_{param, CO_2, x}$: Parameterisation reference data heating rates, 
     ten values at each scale height corresponding to different CO$_2$ column 
     amounts
   \end{enumerate}

   \paragraph{Outputs}
   \begin{enumerate}
   \item $q_{CO_2, unscl}$: CO$_2$ heating rate in K/s corresponding to zenith path
   \end{enumerate}

   \paragraph{Requirements}
   \begin{itemize}   
   \item Spline interpolation code - is it okay to use interpolate\_p?
   \end{itemize}

\subsection{Scale the heating rate}

   In the UM:
   \begin{itemize}
   \item The UM converts fluxes to increments, and then multiplies
     by $\frac{pts}{C_P}$ where $pts$ is the timestep factor, and 
     $C_P$ the layer heat capacity which appears to be per pressure 
     (looking at the subroutine set\_thermodynamic).
   \item Before passing this heating rate into the NLTE correction 
     code, it is multiplied by~$\frac{s/day}{pts}$. This 
     converts the heating rate into the form~$\frac{Q(in K/day)}{C_P}$.
   \end{itemize}

   In the correction code:
   \begin{itemize}
   \item In the NLTE correction code, the heating rate found from the
     parameterisation is in units $\frac{Q(in K/s)}{C_V c_{CO_2}}$ . 
   \item This is multipled by $(s/day) c_{CO_2}(C_V/C_P)$ to convert to 
     the same form as the UM heating rates.
   \end{itemize}

   For an ideal gas

   \noindent $C_P = C_V + nR$

   \noindent For a monotonic gas

   \noindent $C_V = \frac{3}{2} nR$ so $C_P = \frac{5}{2} nR$ and $C_V/C_P = \frac{3}{5}$ 
   
   \noindent For a polyatomic gas
   
   \noindent $C_V = \frac{5}{2} nR$ so $C_P = \frac{7}{2} nR$ and $C_V/C_P = \frac{5}{7}$ \\

   These two cases are blended above $x \geq 10.5$ where atomic oxygen
   makes a difference, as with the heat capacity for the LW NLTE
   correction. So \\
   For  $x < 10.5$:

   \noindent $C_V/C_P = \frac{5}{7}$

   \noindent For $x >= 10.5$ :
   
   \noindent $C_V/C_P = \frac{5}{7} (1 - vmr_{o,x}) + \frac{3}{5} vmr_{o,x}$

   The scaled volume mixing ratio is calculated according to the 
   tropospheric values as

   \noindent $c_{CO_2, x, scl} = c_{CO_2, x} \times \frac{c_{CO_2, x=trop}}{c_{CO_2, x=trop, ref}}$
   
   \noindent And the overall scaling

   \noindent $q_{CO_2} = q_{CO_2, unscl} \times c_{CO_2, x, scl} (C_V/C_P) (s/day)$

   \paragraph{Inputs}
   \begin{enumerate}
   \item $c_{CO_2, x}$: CO$_2$ volume mixing ratio
   \item $c_{CO_2, x=trop, ref}$: Tropospheric value of CO$_2$ volume mixing ratio from parameterisation data 
   \item $vmr_{o,x}$: Volume mixing ratio of atomic oxygen per scale height
   \item $q_{CO_2, unscl}$: CO$_2$ heating rate in K/s corresponding to zenith path
   \end{enumerate}

   \paragraph{Outputs}
   \begin{enumerate}
   \item $q_{CO_2}$: CO$_2$ heating rate in K/day corresponding to zenith path 
   \end{enumerate}

\subsection{Outstanding Issues}

\begin{enumerate}
\item The $\cos$ of the solar zenith angle used was not the correct 
  version - the spherical geometry version is needed, but trying 
  to put this in makes things crash!
\item Need to finish calculating the $O_3$ NLTE profile to add to the 
  $CO_2$ NLTE profile. This needs using the per-band flux up and down
  rates to separate out band 1 heating rate (this can be used for the 
  Hartley band, which takes the NLTE efficiency correction) and bands
  2-4 can be used as a proxy for the Huggins and Chappuis $O_3$ heating 
  rates. 
\item The $O_3$ Hartley efficiency function has a rough part which 
  needs a bit of interpolation over. 
\end{enumerate}
